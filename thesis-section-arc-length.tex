
\section{Localization of Sensitivity in Phase Space}

These results follow from the results of Shaw et al 2012 (see Appendix \ref{iris_summary} for a brief summary of results).  When viewing the iPRC in terms of phase, we gain insight into several useful qualitative behaviors: first, we have numerical evidence (and proof in the case of the iris system) that peak sensitivity increases in the limit of a homoclinic (heteroclinic) bifurcation for an ever-decreasing range of phase values.  Second, in the limit of the homoclinic (heteroclinic) bifurcation, there is an ever-increasing range of phase values in which the sensitivity remains relatively small (Fig.~\ref{fig:iprcs-phase}).  We can describe these behaviors fully in the iris system.  In order to better understand the changing shape of the iPRC in the limit of the homoclinic (heteroclinic) bifurcation, we opt to view the iPRC of each system in a different light.  When we re-plot the iPRCs of the systems under consideration, we find that all iPRCs share a common trait: the iPRCs appear to converge (numerically) to a kink in the 
limit of the homoclinic (heteroclinic) 
bifurcation.  This kink indicates that there is a sudden change in 
sensitivity of the limit cycle. 


%When viewing the iPRC in terms of phase for the iris system, we gain insight into several useful qualitative behaviors: first, we have numerical evidence (and proof in the case of the iris system) that peak sensitivity increases in the limit of a homoclinic (heteroclinic) bifurcation for ever-decreasing phase values.  Second, in the limit of the homoclinic (heteroclinic) bifurcation, there are ever-increasing values of phase in which the sensitivity becomes vanishingly small (Fig.~\ref{fig:iprcs-phase}).  We describe  these behaviors fully in the iris system (Shaw et al. 2012).  In order to better understand the changing shape of the iPRC in the limit of the homoclinic (heteroclinic) bifurcation, we opt to view the iPRC of the iris system in a different light.  When we re-plot the iPRC in terms of arc length, we find that the iPRC appears to converge (numerically) to a corner in the limit of the heteroclinic bifurcation.  This corner, or ``kink'' indicates that there is a sudden change in sensitivity of the 
%limit cycle. 



\begin{figure}
 
 \caption{Show iris numerical iPRC}%All systems under consideration share a distinct feature - the region of the iPRC corresponding to the saddle point appears increasingly corner-like in the limit of the homoclinic (heteroclinic) bifurcation.}
\label{fig:iprcs-corner}\end{figure}

Even more useful and interesting is that the corner of the iPRC of all systems correspond almost exactly to perturbations in the neighborhood of the (a) saddle.  Here, I provide a proof for why this rapids change in sensitivity exists in the special case of the iris system.% and why the change becomes more pronounced and corner-like in the limit of the heteroclinic bifurcation.


\subsection{Proof of Convergence to the Kink for the Iris System}
In order to show that the kink exists for the iPRC of the iris system in the limit of the heteroclinic bifurcation, we first need  a transformation from phase to arc length.  It is virtually impossible to find an exact map for this transformation, but there are some properties of the arc length function which we can use to our advantage.  First, we can prove that the arc length function $L: \phi \mapsto L(\phi)$ converges pointwise to a step function.  Close to the heteroclinic bifurcation, a piecewise linear approximation to the arc length function $L$ is then a valid choice to describe the interplay between arc length and phase.  We then use this piecewise linear approximation to show that for most perturbations prior to the saddle result in negligible phase change, while perturbations in a small neighborhood of the saddle and beyond result in a much greater phase change.

To prove pointwise convergence to the step function, I show that all phases $\phi \in (0,1)$ converge to arc length unity.  The arc length value for phase $\phi = 0$ follows naturally from the definition.

\subsubsection{Proof of convergence to arc length unity}
%Here I show that given an arbitrary phase $\hat\phi \in (0,1)$, its corresponding points in the iris square converge to the saddle point in every square of the iris system in the limit of the heteroclinic bifurcation.  This proves that the arc length function converges to a step function.  

I claim that given any small $\varepsilon>0$, we are guaranteed to find an entry coordinate, $u$, within an $\varepsilon$-ball of the saddle point, which proves that the arc length of any phase converges to unity in the limit of the heteroclinic bifurcation. \textit{Note: All trajectories entering an iris square do so at the point $(1,u)$.  Since the abscissa is known and fixed, I only need to consider the ordinate $u$}.

%I first derive the equation for phase in terms of arbitrary coordinates $(x,y) \in (0,1)\times (0,1)$, then derive equations of $x$ in terms of $y$ (and vice-versa) for some arbitrary, fixed phase value $\hat\phi$. 

%I will conclude with a property of Eq.~\eqref{phixy} consistent with our previous paper, namely that the phase converges to phase $\frac{1}{1+\lambda}$ (the phase closest to the saddle point and the phase of the slowest velocity in the limit of the bifurcation) as $x \rightarrow 0$ and $y \rightarrow 0$.\\

\begin{lemma}Let $x$ and $y$ be arbitrary coordinates in the Southwest iris square. The equation for phase $\phi$ in terms of $x$ and $y$ is
\begin{equation}\label{phixy}
\phi(x,y) = \frac{\log(x)}{\lambda \log(y x^{1/\lambda})}.
\end{equation}
\label{lemma:phi-xy}\end{lemma}
\textit{Proof of Lemma \ref{lemma:phi-xy}}

Define
\begin{equation}
x = s(t_*) = s_0 e^{-\lambda t_*},
\end{equation}

where $x$ is given and we wish to solve for $t_*$, the time at which we reach $x$.  The initial coordinate $s_0$ is equal to unity.  Applying the log function yields an equation for $t_*$ in terms of $x$:
\begin{equation}\label{tstar}
t_* = \frac{\log(x)}{-\lambda}.
\end{equation}

Dividing by the formula for total dwell time in an iris square yields
\begin{equation}\label{phixytemp}
\begin{split}
\phi(x,y) &=\frac{\log(x)}{-\lambda \log(1/u(x,y))}\\
&=\frac{\log(x)}{\lambda \log(u(x,y))},\\
\end{split}
\end{equation}

where $u$ is the initial entry coordinate in terms of $x$ and $y$.  We can now use \eqref{tstar} to solve for $u$ in terms of $x$ and $y$.
\begin{equation}
\begin{split}\label{uxy}
&y = u(t_*) = u e^{t_*},\\
&\Leftrightarrow u = y e^{-t_*},\\
&\Leftrightarrow u = y \exp{\left (\frac{\log(x)}{\lambda} \right )},\\
&\Leftrightarrow u = y \left [\exp{(\log(x))} \right ]^{1/\lambda},\\
&\Leftrightarrow u = y x^{1/\lambda}.
\end{split}
\end{equation}

Combining Eq.~\eqref{uxy} and Eq.~\eqref{phixytemp} completes the proof of lemma \ref{lemma:phi-xy}. $\Box$

The total distance from the right edge of an iris square to the left edge of the same square is of unit length. Moreover, taking the limit as $u \rightarrow 0$ is equivalent to taking the limit of the heteroclinic bifurcation in the iris system, since for each small entry coordinate $u > 0$, there exists a bifurcation parameter\footnote{??? isn't the bifurcation at $a=0$?} value $a>0$, and $u = a + o(u)$ (Shaw et al. 2012). Therefore, all that remains to show is that for a fixed phase value $\hat{\phi}$, the points $x$ and $y$ that correspond to $\hat{\phi}$ approach the saddle point as I take the limit $u \rightarrow 0$.
%Note that Eq.~\eqref{uxy} serves as an equation for the entry coordinate in terms of the arbitrary coordinates $x$ and $y$.  I now show that for any arbitrary phase $\hat\phi \in (0,1)$, $(x,y) \rightarrow (0,0)$ as $u \rightarrow 0$.\\

\begin{lemma}
 $\forall \varepsilon > 0$, $\exists u >0$ s.t. $||(x,y)||_1 < 2\varepsilon$.  
\label{lemma:arc-len-1-convergence}\end{lemma}
\textit{Proof of lemma \ref{lemma:arc-len-1-convergence}}: Let $\hat\phi$ be any fixed value in the open interval $(0,1)$.
\begin{equation}
\begin{split}
&\hat\phi = \frac{\log(x)}{\lambda \log(y x^{1/\lambda})}\\
&\Leftrightarrow\lambda \hat\phi \log(yx^{1/\lambda}) = \log(x)\\
&\Leftrightarrow (y x^{1/\lambda})^{\lambda \hat\phi} = x\\
&\Leftrightarrow y^{\lambda \hat\phi} x^{\hat\phi} = x\\
&\Leftrightarrow y^{\lambda \hat\phi}  =x^{1-\hat\phi}\\
&\Leftrightarrow y  =x^{\frac{1}{\lambda}(\frac{1}{\hat\phi} - 1)}.\\
\end{split}
\end{equation}

%Note that the exponent of $x$ ranges from arbitrarily small (and positive) to arbitrarily large (and positive) depending on the value of $\hat\phi$.  This is the reason why the level curves of Eq.~\eqref{phixy} form an onion shape.

We now use Eq.~\eqref{uxy} to find a relationship between $u$ and $x$.% (likewise, it is easy to derive a relationship between $u$ and $y$, but I will leave this as an exercise for the reader).
\begin{equation}
\begin{split}
&u = y x^{1/\lambda}\\
&\Leftrightarrow y = u x^{-1/\lambda}\\
&\Leftrightarrow  u x^{-1/\lambda} = x^{\frac{1}{\lambda}(\frac{1}{\hat\phi} - 1)}\\
&\Leftrightarrow  u = x^{\frac{1}{\lambda \hat\phi}}.\\
\end{split}
\label{eq:u-x}\end{equation}

We can also find a relationship between $u$ and $y$:
\begin{equation}
\begin{split}
&u = y x^{1/\lambda}\\
&\Leftrightarrow x = \left ( \frac{u}{y} \right ) ^\lambda \\
&\Leftrightarrow  \left ( \frac{u}{y} \right ) ^\lambda = y^{\frac{\lambda}{(1/\hat{\phi}-1)}}\\
&\Leftrightarrow  u^\lambda = y^{\lambda \left (\frac{1}{1/\hat{\phi}-1} +1\right )}\\
&\Leftrightarrow  u = y^{\frac{1}{1-\hat{\phi}}}.
\end{split}
\label{eq:u-y}\end{equation}

In order to show convergence, let $1 \gg \varepsilon > 0$ be given.  Define $u = \varepsilon^{1/(\hat{\phi}(1-\hat{\phi}))} \leq \varepsilon^4 < \varepsilon$.  By Eqs.~\eqref{eq:u-x} and \eqref{eq:u-y}, we have
\begin{equation}
\begin{split}
  |x| + |y| &= |u^{\lambda \hat{\phi}}| + |u^{1-\hat{\phi}}| \\
  &= \varepsilon^{\lambda/(1-\hat{\phi})} + \varepsilon^{1/\hat{\phi}} \\
  &< \varepsilon + \varepsilon = 2\varepsilon.
\end{split}
\end{equation}

This concludes the proof that the arc length of all phases $\hat{\phi} \in (0,1)$ converge to 1 in the limit of the heteroclinic bifurcation. $\Box$

\begin{figure}[h!]
 \caption[Geometric interpretation of phase convergence]{Geometric interpretation of phase convergence.}
\label{fig:cartoon-convergence-step-fn}\end{figure}


%Recall:
%\begin{equation}
%\begin{split}
%L &= \int_0^{\phi T} \! \sqrt{(-\lambda e^{-\lambda s})^2 + (u e^s)^2} \, \mathrm{d}s\\
%\Leftrightarrow \frac{d L}{d \phi} &= \frac{d}{d \phi }\int_0^{\phi T} \! \sqrt{(-\lambda e^{-\lambda s})^2 + (u e^s)^2} \, \mathrm{d}s\\
% &=T \sqrt{(-\lambda e^{-\lambda \phi T})^2 + (u e^{\phi T})^2}\\
%  &=\log(1/u) \sqrt{\lambda^2  u^{2\lambda \phi} + u^{2(1-\phi)}}\\
%  &=\log(1/u) u \sqrt{\lambda^2  u^{2\lambda \phi-1} + u^{-\phi}}.\\
%\end{split}
%\end{equation}

%Given $\varepsilon > 0$,  Since the derivative of $L(\phi)$ approaches zero for all $\phi \in (0,1)$, we have (pointwise) convergence. Proof!

%The arc length function is both strictly increasing with one point of inflection.  Therefore, the the lines with slopes defined by phases $0$, $\frac{\log(\frac{u^2}{\lambda^3})}{2(1+\lambda)\log(u)}$, and $1$ must meet up to create a piecewise linear caricature of the arc length function (in terms of phase).  Need more proof(s). 

\subsubsection{Division into three distinct regions}
Now I use a piecewise linear approximation (PWLA) to the arc length function to derive an expression for phase in terms of arc length.  To simplify the calculations, I continue to limit the analysis of this section to the iPRC derived from vertical perturbations in the Southwest square.

Since the PWLA is linear, it is relatively easy to find $L^{-1}$, the inverse of the arc length function.  We use $L^{-1}$ to show that past the saddle point, values of the iPRC converge to zero in the limit as the entry point of the square goes to zero.  The value of the iPRC diverges for arc length values in the open unit interval $(0,1)$.  In other words, in the limit of the heteroclinic bifurcation, there is a near-instantaneous transition from the entry coordinate $(1,u)$ to the saddle point.  Once a trajectory reaches the saddle, it remains at the saddle for a long time.  After a long enough time, the trajectory escapes and there is another near-instantaneous transition from the saddle to the exit coordinate $(s,1)$.  The time of flight ($T = \log(1/u)$) in one iris square diverges in the limit, but as long as we are not at the heteroclinic bifurcation, the time of flight remains finite and the description above remains true.

I split the arc length function into three qualitatively distinct regions: the region of initial rise; region of ``convergence'' (where the phases converge to arc length 1 about the phase where the derivative of arc length with respect to phase is the lowest); and region of final rise (Fig.~\ref{arclengthregions}).  The justification for this is that it is easy to find the derivative at phases $\phi = 0, \frac{\log(\frac{u^2}{\lambda^3})}{2(1+\lambda)\log(u)}, 1$ as well the corresponding arc length values ($L = 0, 1, 2$).  Using this information, we may formulate three linear equations (which we will call $L_1, L_2, L_3$) and enforce continuity by solving for the intersections between each line.  I will denote boundary values of $\phi$ and $L$ by the notation $\overline\phi_{ij}$ and $\overline L_{ij}$ where $ij$ denotes the $i^{th}$ and $j^{th}$ regions, respectively.  I later derive the values of the boundaries in the limit.

\begin{figure}[h!]
\begin{center}
\includegraphics[width=0.8\textwidth]{three_regions.png}
\caption[Division of the arc length function into three regions]{Division of the arc length function into three regions.  The three qualitatively distinct regions are labeled as above.  The parameter $\lambda$ was chosen to take the value $2$, and the initial entry coordinate $u$ was chosen to be $\approx 10^{-50}$).}\label{arclengthregions}
\end{center}
\end{figure}

\begin{figure}[h!]
 \caption{The piecewise linear regions link up}
\end{figure}

It makes sense to define the middle piece such that the slope is the minimum derivative of the function $L(\phi)$.  We can find this phase value by the usual method -- by finding the critical point of $\frac{dL(\phi)}{d\phi}$.  The critical point is at the phase

\begin{equation}
 \phi_c = \frac{\log(\frac{u^2}{\lambda^3})}{2(1+\lambda)\log(u)}
\end{equation}


%\begin{equation}
% \frac{d^2L}{d\phi^2} = \frac{u^{-2\phi} \log(1/u) \log(u) (-u^2 + 2u^{2(1+\lambda)\phi}\lambda^{(1+2u^})}{}
%\end{equation}




\subsubsection{Derivatives of the arc length function with respect to phase and evaluated at $\phi = 0,  \phi_c, 1$}

Recall:
\begin{equation}
 L(\phi) = \int_0^{\phi T} \! \sqrt{(-\lambda e^{-\lambda s})^2 + (u e^s)^2} \, \mathrm{d}s,
\end{equation}

and 
\begin{equation}
 \frac{d L}{d \phi} = \log(1/u) \sqrt{\lambda^2  u^{2\lambda \phi} + u^{2(1-\phi)}}.
\end{equation}


We can evaluate $\frac{dL}{d\phi}$ at the chosen phase values to get the following derivatives:
\begin{align}
&\frac{d L}{d \phi}|_{\phi=0} = \log(1/u)\sqrt{\lambda^2+u^2}=a_1,\\
&\frac{d L}{d \phi}|_{\phi=\phi_c} = \log(1/u) \sqrt{\lambda^2  u^{2\lambda \phi_c} + u^{2(1-\phi_c)}} = a_2,\\ %\log(1/u)\sqrt{u(\lambda^2+1)}=a_2,\\
&\frac{d L}{d \phi}|_{\phi=1} = \log(1/u)\sqrt{\lambda^2 u^{2\lambda}+1}=a_3,
\end{align}

so $0 < a_2 \ll 1 \ll a_1,a_3$.  Convergence of $a_2$ follows because for any $\phi < 1/2$

\begin{equation}
 \log(1/u)\sqrt{\lambda^2 u^{2 \lambda \phi} + u^{2(1-\phi)}} \leq \log(1/u) u^{(\phi/2)} \sqrt{\lambda^2 + 1}
\end{equation}


\subsubsection{Piecewise linear arc length}

%\pjt{Just to clarify: you are going to say that arc length is a monotonically and smoothly increasing (for $u>0$) function of phase, $L(\phi)$, which you are going to approximate with a piecewise linear increasing function defined below. -PJT}

Solving for the linear equations for each region is a straightforward exercise.  Assuming that we are close enough to the bifurcation to let $L(1) = 2 + O(u)$ and recalling that $L(0)=0, L(\phi_c)=1 + O(u)$,
\begin{align}
 &L_1(\phi) = a_1 \phi, \, \, \phi \in [0, \overline \phi_{12}),\\
 &L_2(\phi) = a_2 \phi + (1-\frac{a_2}{2}) + O(u), \, \, \phi \in [\overline \phi_{12}, \overline \phi_{23}],\\\
 &L_3(\phi) = a_3 \phi + 2-a_3 + O(u), \, \, \phi \in ( \overline \phi_{23}, 1].
\end{align}

%\pjt{You don't state this, but I assume you are going to have $0< (a_1,a_3)\ll 1 \ll a_2<\infty$, correct?  Also, since $L(1)<2$ when $u>0$, do you know how close $L(1)$ is to 2?  Can you say, e.g., $L(1)=2-b_2u^p+o(u^p)$ for some power $p$ and some constant $b_2$?  And is $L(1/2)=1-b_1 u^q+o(u^q)$ for some power $q$ (maybe the same as $p$) and some constant $b_1$?  Perhaps pinning down these approximations doesn't matter much at this point. -PJT }

We can now solve for phase $\phi$ in terms of arc length $L$
\begin{align}
 &\phi_1(L) = \frac{L}{a_1}, \, \, L \in [0,\overline L_{12}),\\
 &\phi_2(L) = \frac{L-1}{a_2}+\frac{1}{2}+ O(u), \, \, L \in [\overline L_{12}, \overline L_{23}],\\
 &\phi_3(L) = \frac{L-2}{a_3}+1 + O(u), \, \, L \in (\overline L_{23},2].
\end{align}


Given the equations above, we can also solve for the boundaries $\overline\phi_{ij}$ and $\overline L_{ij}$.
\begin{align}
 &\overline\phi_{12} = (1-\frac{a_2}{2})/(a_1-a_2) + O(u),\\
 &\overline\phi_{23} = (1+\frac{a_2}{2}-a_3)/(a_2-a_3) + O(u),\\
 &\overline L_{12} = \left (1-\frac{a_2}{2} \right ) \left ( \frac{a_1}{a_1-a_2} \right) + O(u),\\
 &\overline L_{23} = a_3 \left ( \frac{1-\frac{a_2}{2}}{a_2-a_3} \right)+2 + O(u)
\end{align}


Note that $\overline L_{ij}$ and $\overline\phi_{ij}$ together constitute the intersections of the lines between regions $i$ and $j$.  See Fig.~\ref{pwla} for a visual representation of the system constructed so far.

\begin{figure}[h!]
\begin{center}
\includegraphics[width=0.8\textwidth]{pwla.png}
\caption{Qualitative comparison of PWLA to original arc length equation.  Red: Piecewise linear equations; Blue: Cartoon of original arc length function relative to piecewise equations.  Boundaries in $\phi$ are separated by vertical dashed lines.}\label{pwla}
\end{center}
\end{figure}

\subsubsection{Converge and divergence of the iPRC in terms of arc length}

We now consider the limit of the iPRC in the limit of the heteroclinic bifurcation in terms of arc length.  Recall that the iPRC for perturbations in the vertical direction in the first iris square is defined as
\begin{equation}
 Z_y(\phi) = \frac{u^{\phi}}{\log(1/u) u (1-\lambda u^{\lambda-1})}.
\end{equation}


\textit{Consider region 1}:

By using the phase to arc length transformation, it is possible to write the iPRC in terms of arc length.
\begin{equation}
\begin{split}
 Z_{y,1}(\phi) &= Z_1(L/a_1)\\
 &=\frac{u^{L/a_1}}{\log(1/u) u (1-\lambda u^{\lambda-1})}\\
 &=\frac{u^{L/[\log(1/u)\sqrt{\lambda^2+u^2}]}}{\log(1/u) u (1-\lambda u^{\lambda-1})},\\
 \end{split}
\end{equation}

where we fix $L \in (0,\overline L_{12})$.  So for small $u$,
\begin{equation}
\frac{u^{L/[\log(1/u)\sqrt{\lambda^2+u^2}]}}{\log(1/u) u (1-\lambda u^{\lambda})} \geq \frac{1}{\log(1/u) u^{1/2} (1-\lambda u^{\lambda-1})}.
\end{equation}

The right-hand side diverges as $u \rightarrow 0$, so by the comparison test, the left-hand side diverges as well.  We see here that any $L \in [0,\overline L_{12})$ plays no role in the limit.  It may be worth checking that $\overline\phi_{12} \rightarrow 0$ as $u \rightarrow 0$.


\textit{Consider region 2}:
\begin{equation}
\begin{split}
 Z_{y,2}(\phi) &= Z_2\left (\frac{L-1}{a_2}+\frac{1}{2} \right) \\
 &= \frac{u^{\left ( \frac{L-1}{a_2}+\frac{1}{2} \right )}}{\log(1/u) u (1-\lambda u^{\lambda-1})},
\end{split}
\end{equation}

where we fix $L \in [\overline L_{12}, \overline L_{23}]$.  In the limit as $u \rightarrow 0$, $\overline L_{12}, \overline L_{23} \rightarrow 1$
 
\begin{equation}
 \frac{u^{\left ( \frac{L-1}{a_2}+\frac{1}{2} \right )}}{\log(1/u) u (1-\lambda u^{\lambda-1})} \approx \frac{1}{\log(1/u) u^{1/2}},
\end{equation}

which also diverges as $u \rightarrow 0$.

\textit{Consider region 3}:

For this region, fix $L \in (\overline L_{23},2)$.

\begin{equation}
\begin{split}
 Z_{y,3}(\phi) &= Z_3\left (\frac{L-2}{a_3}+1 \right) \\
 &= \frac{u^{\left ( \frac{L-2}{a_3}+1 \right )}}{\log(1/u) u (1-\lambda u^{\lambda-1})}\\
 &\leq  \frac{u}{\log(1/u) u (1-\lambda u^{\lambda-1})}\\
 &=  \frac{1}{\log(1/u) (1-\lambda u^{\lambda-1})}.\\
\end{split}
\end{equation}

The last line converges to zero as $u \rightarrow 0$, so by the comparison test, $Z_3$ converges to zero as well.  Notice that again, the convergence works for any arc length $L \in (\overline L_{23},2]$, even though $\overline\phi_{23} \rightarrow 1$ as $u \rightarrow 0$.

From these three cases, we can conclude that for arc length $L \in [0,1]$, the iPRC diverges, but for arc length $L \in (1,2]$, the iPRC converges to zero.  I believe that the PWLA works because both functions converge to the step function in the limit of the bifurcation.  Proving this will need careful treatment.

\subsubsection{Convergence of boundary values}

\begin{equation}
 \overline L_{12} = \left (1 - \frac{1}{2} \log(1/u) \sqrt{u (\lambda^2 + 1)} \right ) \left ( \frac{\log(1/u) \sqrt{\lambda^2 + u^2}}{\log(1/u) \sqrt{\lambda^2 + u^2} - 1 + \frac{1}{2} \log(1/u) \sqrt{u (\lambda^2 + 1) }} \right ).
\end{equation}

In the limit as $u \rightarrow 0$, the term $ \log(1/u) \sqrt{u (\lambda^2 + 1)}$ converges to zero , while the term $\log(1/u) \sqrt{\lambda^2 + u^2}$ diverges.  Therefore, $\overline L_{12} \rightarrow 1$ as $u \rightarrow 0$.  The other boundary value, $\overline L_{23}$, converges in a similar way:
\begin{equation}
\begin{split}
 \overline L_{23} &=\log(1/u) \sqrt{\lambda^2 u^{2\lambda} + 1} \left ( \frac{1-\frac{1}{2}\log(1/u) \sqrt{u(\lambda^2 + 1)}}{\log(1/u) \sqrt{u(\lambda^2 + 1)} - \log(1/u) \sqrt{\lambda^2 u^{2 \lambda} + 1 }} \right )+2\\
 &= \sqrt{\lambda^2 u^{2\lambda} + 1} \left ( \frac{1-\frac{1}{2}\log(1/u) \sqrt{u(\lambda^2 + 1)}}{ \sqrt{u(\lambda^2 + 1)} -  \sqrt{\lambda^2 u^{2 \lambda} + 1 }} \right )+2\\
 &\approx  \left ( \frac{1}{ -1 } \right )+2=1.
\end{split}
\end{equation}

Now consider the other boundary values:
\begin{equation}
 \overline\phi_{12} = \left(1-\frac{1}{2}\log(1/u)\sqrt{u(\lambda^2+1)} \right )/ \left (\log(1/u)\sqrt{\lambda^2+u^2}-\log(1/u) \sqrt{u(\lambda^2+1)} \right ).
\end{equation}

The bottom diverges without bound, while the top converges to zero.  Therefore $\overline\phi_{12} \rightarrow 0$ as $u \rightarrow 0$.
\begin{equation}
 \overline\phi_{23} = \frac{\left (1+\frac{1}{2} \log(1/u) \sqrt{u(\lambda^2+1)} - \log(1/u) \sqrt{\lambda^2 u^{2 \lambda} + 1} \right)}{\left ( \log(1/u) \sqrt{u(\lambda^2+1)} - \log(1/u)\sqrt{\lambda^2 u^{2\lambda} + 1}\right)}.
\end{equation}

The terms $\log(1/u) \sqrt{u(\lambda^2+1)}$ converge to zero, while the terms $\log(1/u)\sqrt{\lambda^2 u^{2\lambda} + 1}$ diverge without bound.  Therefore $\overline\phi_{23} \rightarrow 1$ as $u \rightarrow 0$.



%\subsection{Numerical Evidence That the Kink Exists in General}
%\begin{itemize}
% \item 
%\end{itemize}


\subsection{Tentative Generalization to Homoclinic Systems}

\begin{itemize}
 \item transform coordinates
 \item translate origin to zero
 \item trajectories must decompose properly into two exponentials
\end{itemize}

